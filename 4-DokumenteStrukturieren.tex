\documentclass[ 12pt, a4paper, parskip=full]{scrartcl}

% Die Gestaltung der Absätze ist im deutschen Sprachraum   
% anders als im englischen Sprachraum. Die Anpassung an
% den deutschen Sprachraum erreichen Sie mit: parskip=full.

% Für eine vollständige Anpassung an die deutsche Sprache;
\usepackage[utf8]{inputenc}   % für deutsche Umlaute
\usepackage[ngerman]{babel}   % für neue deutsche Rechtschreibung 
\usepackage[T1]{fontenc}      % für Umlaute im pdf-Dokument


\begin{document}

% Kapitel
\section{Herkunft}

Das Basis-Programm von LaTeX ist TeX und wurde von Donald E. Knuth während seiner Zeit als Informatik-Professor an der Stanford University entwickelt. Auf TeX aufbauend entwickelte Leslie Lamport Anfang der 1980er Jahre LaTeX, eine Sammlung von TeX-Makros, die die Benutzung für den durchschnittlichen Anwender gegenüber TeX vereinfachten und erweiterten. Der Name LaTeX ist eine Abkürzung für Lamport TeX. 

% Kapitel
\section{WYSIWYG vs WYSIWYAF}

% Unterkapitel
\subsection{WYSIWYG}
Im Gegensatz zu anderen Textverarbeitungsprogrammen, die nach dem WYSIWYG (What-you-see-is-what-you-get) Prinzip funktionieren, arbeitet der Autor mit Textdateien, in denen er innerhalb eines Textes anders zu formatierende Passagen oder Überschriften mit Befehlen textuell auszeichnet.

% Unterkapitel
\subsection{WYSIWYAF}
Das dabei von LaTeX generierte Layout gilt als sehr sauber, sein Formelsatz als sehr ausgereift. Außerdem ist die Ausgabe u. a. nach PDF, HTML und PostScript möglich. LaTeX eignet sich insbesondere für umfangreiche Arbeiten wie Diplomarbeiten und Dissertationen, die oftmals strengen typographischen Ansprüchen genügen müssen. Insbesondere in der Mathematik und den Naturwissenschaften erleichtert LaTeX das Anfertigen von Schriftstücken durch seine komfortablen Möglichkeiten der Formelsetzung gegenüber üblichen Textverarbeitungssystemen. Das Verfahren von LaTeX wird auch mit WYSIWYAF (What you see is what you asked for.) umschrieben. 

Wie TeX selbst ist LaTeX weitestgehend rechnerunabhängig verwendbar. Das bedeutet, dass es für die meisten Betriebssysteme analog zu TeX auch für LaTeX lauffähige, produktiv einsetzbare LaTeX-Installationen gibt. Zu diesen Betriebssystemen gehören zum Beispiel Microsoft Windows von der Version 3.x bis zur aktuellen Version 10, Apple macOS sowie diverse Linux-Distributionen. Unter der Voraussetzung, dass alle verwendeten Zusatzpakete (siehe unten) in geeigneten Versionen installiert sind, besteht der Vorteil der Verwendung von LaTeX darin, dass das Ergebnis unabhängig von der verwendeten Rechnerplattform und dem verwendeten Drucker in den beiden Ausgabeformaten DVI und PDF hinsichtlich der Schriftarten und -größen sowie der Zeilen- und Seitenumbrüche im Druck stets exakt gleich ist. LaTeX ist hierbei nicht auf die Schriftarten des jeweiligen Betriebssystems angewiesen. Jene Betriebssystem-Schriftarten sind häufig für die Anzeige am Monitor optimiert. LaTeX enthält eine Reihe eigener Schriftarten, die ihrerseits für den Druck optimiert sind. 

\end{document}
