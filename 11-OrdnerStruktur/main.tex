\documentclass[12pt, a4paper]{article}

\usepackage[utf8]{inputenc}   % für deutsche Umlaute
\usepackage[ngerman]{babel}   % für neue deutsche Rechtschreibung 
\usepackage[T1]{fontenc}      % für Umlaute im pdf-Dokument

\usepackage{amsmath}          % für mathematische Formeln
\usepackage{amssymb}          % für mathematische Symbole

\usepackage{graphicx}         % für Bilder

\setlength{\parindent}{0em}   % Einzug (Einrückung) der ersten Zeile im Abschnitt
\setlength{\parskip}{0.75em}  % Abstand zwischen den einzelnen Abschnitten

\title{\LaTeX-Intro}          % Titel der Arbeit
\author{Peter Kessler}        % Autor der Arbeit
\date{März 2019}              % Datum der Arbeit


\begin{document}

\maketitle

\pagebreak
\tableofcontents



\pagebreak 
\section{Herkunft}
Das Basis-Programm von LaTeX ist TeX und wurde von Donald E. Knuth während seiner Zeit als Informatik-Professor an der Stanford University entwickelt. Auf TeX aufbauend entwickelte Leslie Lamport Anfang der 1980er Jahre LaTeX, eine Sammlung von TeX-Makros, die die Benutzung für den durchschnittlichen Anwender gegenüber TeX vereinfachten und erweiterten. Der Name LaTeX ist eine Abkürzung für Lamport TeX. 

\begin{figure}[h!]
\centering
  \includegraphics[width=0.5\textwidth]{./Bilder/overleaf.png}
  \caption{Overleaf - \LaTeX in der Cloud}
\end{figure}



\pagebreak
\section{WYSIWYG vs WYSIWYAF}

\subsection{WYSIWYG}
\input{./Inhalte/WYSIWYG.tex}

\subsection{WYSIWYAF}
\input{./Inhalte/WYSIWYAF.tex}



\pagebreak
\section{Arbeiten mit mathematischen Formeln}
\input{./Inhalte/Math.tex}

\subsection{Inline Formeln}
\input{./Inhalte/MathInline.tex}

\subsection{Abgesetzte Formeln}
\input{./Inhalte/MathAbgesetzt.tex}

\subsection{Ausgerichtete Formeln}
\input{./Inhalte/MathAusgerichtet.tex}

\end{document}
