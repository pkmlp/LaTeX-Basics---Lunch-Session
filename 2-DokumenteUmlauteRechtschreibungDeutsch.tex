\documentclass[12pt, a4paper]{scrartcl}
% In diesem Dokument ist die "Normale-Schriftgrösse"  
% auf 12 Punkte und das Papierformat auf DIN A4 gesetzt
% Als Documentclass wird scrartcl (Koma-Script) verwendet. 
% Die Koma-Script Klassen sind im deutschsprachigen Raum 
% sehr beliebt, da sie sich an den DIN-Normen orientieren. 

% Für eine vollständige Anpassung an die deutsche Sprache;
\usepackage[utf8]{inputenc} % für deutsche Umlaute
\usepackage[ngerman]{babel} % für neue deutsche Rechtschreibung 
\usepackage[T1]{fontenc}    % für Umlaute im pdf-Dokument


\begin{document}


Hello World, hello \LaTeX

\textbf{Hello World, hello \LaTeX}

\textit{Hello World, hello \LaTeX}

\textbf{Hallo schöne \LaTeX Welt}

Und dann ist da noch das Problem mit den deutschen Umlauten ä, ö, ü, Ä. Ö, Ü und der neuen Rechtschreibung im Deutschen.


\end{document}
